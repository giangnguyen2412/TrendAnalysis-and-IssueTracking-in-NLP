\section{Relevant Work}
Topic detection and tracking is not a new field, and the domain has mostly been on news sources. Since NIST first proposed the problem of Topic Detection and Tracking (TDT) in the 1990s \cite{graff1999tdt}, a lot of work has been done. As the first effort, researchers proposed many approaches to detect and track topics on news documents \cite{lavrenko2002relevance} \cite{larkey2004language}. Most of these methods are based on either vector space models \cite{yang1999learning} or statistical models \cite{lavrenko2002relevance} \cite{larkey2004language}. For example, Yang et al. \cite{yang1999learning} represented news documents as vectors of words weighted by term-frequency inverse-document-frequency (TF-IDF) and used cosine angle to measure their similarity. Larkey et al.  \cite{larkey2004language} estimated news documents’ relevance language model and measured their similarity based on the asymmetric clarity-adjusted divergence. However, there is no common model providing state-of-the-art results because of ambiguity, untrustworthiness and redundancy of data. This work should explore merits of modern tools in NLP to build a model that mines media news for valuable information the aforementioned domain. 